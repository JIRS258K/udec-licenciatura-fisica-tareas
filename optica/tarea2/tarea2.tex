\documentclass{article}
\setlength{\headheight}{13.07204pt} % Adjust headheight as suggested
\usepackage[letterpaper, left=2cm, right=2cm, top=2cm, bottom=2cm]{geometry}


% Paquetes matemáticos y tipográficos
\usepackage{mathrsfs}
\usepackage{amssymb}
\usepackage{amsmath}
\usepackage{amsfonts}
\usepackage{mathtools}
\usepackage{booktabs}
\usepackage{siunitx}
\usepackage{graphicx}
\usepackage{booktabs}
\usepackage{caption}
\usepackage{float}      % <— Necesario para [H]
\captionsetup{labelfont=bf}


% Numeración de ecuaciones por sección
\numberwithin{equation}{section}

% Carga paquetes
\usepackage{minted}
\usepackage{xcolor}

% Define colores personalizados
\definecolor{codebg}{gray}{0.95}    % Fondo gris clarito
\definecolor{commentgray}{gray}{0.4} % Comentarios gris medio
\definecolor{keywordgray}{gray}{0.2} % Palabras clave gris oscuro

% Configura minted
\setminted{
    bgcolor=codebg, % Fondo del bloque
    linenos,        % Mostrar números de línea
    breaklines,     % Romper líneas largas
    fontsize=\small,
    style=default,  % Estilo base
}



% Para la terminal
\usepackage{tcolorbox}
\tcbuselibrary{listingsutf8}
\usepackage{listings}
\usepackage{bera} % Fuente monoespaciada bonita (opcional)

% Idioma en español
\usepackage[spanish]{babel}

% Manejo de imágenes
\usepackage{graphicx} 
\graphicspath{ {images/} }

% Configuración de márgenes
\usepackage[letterpaper, left=2cm, right=2cm, top=2cm, bottom=2cm]{geometry}


% Tipografía mejorada
\usepackage{lmodern}

% Estilo de títulos con punto después del número
\usepackage{titlesec}
\titleformat{\section}{\huge\bfseries}{\thesection.}{1em}{}  % Título más grande

% Encabezados sin pie de página
\usepackage{fancyhdr}
\pagestyle{fancy}
\fancyhf{}
\fancyhead[R]{\textit{Óptica matricial}}
\fancyhead[L]{Física V: 510355 - Óptica}

% Mejor separación de párrafos
\setlength{\parindent}{0pt}
\setlength{\parskip}{5pt}

% Evita hifenaciones excesivas
\sloppy

% Configuración del índice
\usepackage{tocloft}
\setcounter{tocdepth}{2}

\begin{document}

% Portada
\begin{titlepage}
    \centering
    \vspace*{3cm} % Ajuste en la posición vertical
    % Logo centrado
    \includegraphics[width=0.6\textwidth]{UdeC_azul_centrado.png} 
    
    \vspace{1cm}
    \thispagestyle{empty} % Sin número en la portada

    % Título de la tarea
    {\Huge \textbf{Tarea 02 - Óptica matricial} \par}
    
    \vspace{0.5cm}
    {\Huge \textbf{Física V: 510355 - Óptica} \par}
    \vspace{1.5cm}

    % Nombre del autor
    {\Large José Ignacio Rosas Sepúlveda \par
    Kevin Andrés Vergara González \par
    Víctor Patricio Maureira Cárdenas}
    \vspace{1cm}
    
    % Fechas de la tarea
    {\Large Junio 2025 \par}
    \vfill
\end{titlepage}

\newpage
\section*{Situación 1}
Un pequeño objeto está dentro de un recipiente de ámbar. El recipiente es una esfera maciza de vidrio de índice de refracción $n' = \frac{8}{5}$, radio de curvatura $|R| = 3\,\text{cm}$ y se encuentra rodeado de aire de índice de refracción $n = 1$. El objeto está sobre el eje óptico y cuando es visto a lo largo del eje, y a través de la superficie, parece estar enterrado $5\,\text{cm}$ dentro del recipiente. Calcule la distancia a la cual el mosquito está realmente dentro de la esfera.

\textbf{Solución:}
\renewcommand{\arraystretch}{1.5}
%%%%%%%%%%%%%%%%%%%%%%%%%%%%%%%%%%%%%%%%%%%%%%%%%%%%%%%%%%%%%%%%%%%%%%%%%%%%%%%%%%%%%%%%%%%%%%%%%%%%%%%%%%%%%%%%%%%%%%%%%%%%%%%%%%%%%%%%%%%%%%%%%%%%%%%%%%%%%%%%%%%%%%%%%%%%%%%%%%%%%%%%%%%%%%%%%%%%%%%

Para mejor entendimiento del desarrollo de esta situación, iremos paso a paso realizándolo en base a la estructura mencionada en la tarea
\subsection*{Paso 1}
Identificamos el sistema optico y la matriz $[ABCD]$, pero antes, identificaremos los datos importantes como:
\begin{itemize}
    \item Indice de refraccion del vidrio: $n_1=\frac{8}{5}$
    \item Indice de refraccion del aire: $n_2=1$
    \item Radio de curvatura de la superficie $R=-3$(dado que es convexo desde la perspectiva del aire)
    \item La imagen aparente se observa a $s'=-5\,[\mathrm{cm}]$ desde la superficie
    \item La luz se propaga de derecha a izquierda
\end{itemize}
Notemos que como el sistema consta de una sola superficie esférica convexa que separa dos medios, usamos la matriz de refracción esférica, que es de la forma:
\begin{align*}
    M_s=&\left[\begin{matrix}
         1 & 0\\ \frac{(n_1-n_2)}{n_2R}& \frac{n_1}{n_2}
    \end{matrix}\right ]\\
    M_s=&\left[\begin{matrix}1&0\\\left(\frac{\frac{8}{5}-1}{1\cdot(-3)}\right)&\frac{8}{5} 
    \end{matrix}\right]\\
\end{align*}
Por lo tanto la matriz [ABCD] está dada por
\begin{align}
    \left[\begin{matrix}
        A&B\\C&D
    \end{matrix}\right]=\left[\begin{matrix}
        1&0\\-\frac{1}{5}&\frac{8}{5}
    \end{matrix}\right]
\end{align}
\subsection*{Paso 2}
Es muy importatnte notar que un punto en el eje optico se representa como un vector, esto nos ayudara a encontrar la distancia a la cual se encuentra el objeto(en este caso el mosquito). Por tanto nuestro vector seria de la forma:
\begin{align*}
    \vec v=\left[\begin{matrix}
        y\\\theta
    \end{matrix}\right]
\end{align*}
Para un objeto puntual sobre el eje optico, usamos su posicion relativa $s$ como parte del sistema de propagacion. Como solo tenemos una superficie usaremos la matriz para vincular la imagen y el objeto.
Notemos que el vector imagen en optica matricial esta dado por:
\begin{align*}
    \vec v'=\left[\begin{matrix}
        0\\\theta'
    \end{matrix}\right]
\end{align*}
Y ya con esto planteado, relacionamos el objeto con:
\begin{align*}
    \vec v'=M\cdot \vec v
\end{align*}
Para un objeto a una distancia $s$, el rayo que parte del objeto y llega perpendicular al eje llega a altura $y$, con un angulo de $\theta=-\frac{y}{s}$. Usando $y=1$(ya que todo es lineal)
\begin{align*}
    \vec v_{obj}&=\left[\begin{matrix}
        1\\
        -\frac{1}{s}
    \end{matrix}\right]\\
    \vec v_{img}&=M\cdot\vec v_{obj}\\
    \vec v_{img}&=\left[\begin{matrix}
        1&0\\-\frac{1}{5}&\frac{8}{5}
    \end{matrix}\right]\cdot\left[\begin{matrix}
        1\\-\frac{1}{s}
    \end{matrix}\right]\\
    \vec v_{img}&=\left[\begin{matrix}
        1\\-\frac{1}{5}-\frac{8}{5s}
    \end{matrix}\right] 
\end{align*}
Notemos que el vector $v_{img}$ es el vector que corresponde al rayo en el aire despues de la refraccion. Como queremos que este rayo forme una imagen virtual a una distancia $s'=-5cm$, el angulo con respecto al eje debe ser el siguiente:
\begin{align*}
    \theta'=-\frac{1}{s'}=\frac{1}{5}
\end{align*}
entonces:
\begin{align*}
    -\frac{1}{5}-\frac{8}{5s}=\frac{1}{5}\implies -\frac{8}{5s}=\frac{1}{5}+\frac{1}{5}\implies -\frac{8}{5s}=\frac{2}{5}\implies -\frac{8}{s}=2
\end{align*}
De esto obtenemos que la posicion real del mosquito es de \boxed{4 cm \text{ dentro del vidrio}}

\subsection*{Paso 3}
Calcularemos los datos de la tabla anteriormente mencionada  
\begin{itemize}
    \item Distancia focal de salida $f'$:
\begin{align*}
    f'&=-\frac{1}{C}=-\frac{1}{-\frac{1}{5}}=\boxed{5cm}\\
\end{align*}
\item Distancia focal de entrada
\begin{align*}
    f=\frac{D}{-C}=\frac{\frac{8}{5}}{-(-\frac{1}{5})}=\boxed{8cm}
\end{align*}
\item Radio de curvatura:
\begin{align*}
    r=R=\boxed{3cm}
\end{align*}
\item Distancia imagen (v):
Relacion matricial para propagacion desde objeto:
\begin{align*}
    \vec v_{in}=\left[\begin{matrix}
        1\\-\frac{1}{s}
    \end{matrix}\right]=\left[\begin{matrix}
        1\\\frac{1}{4}
    \end{matrix}\right]\implies \vec v_{ext}=\left[\begin{matrix}
        1&0\\ -\frac{1}{5}&\frac{8}{5}
    \end{matrix}\right]\cdot\left[\begin{matrix}
        1\\ \frac{1}{4}
    \end{matrix}\right]= \left[\begin{matrix}
        1\\\frac{1}{5}
    \end{matrix}\right]
\end{align*}
Entonces el angulo en el aire es:
\begin{align*}
    \theta = \frac{1}{v}\implies v = \frac{1}{\frac{1}{5}}=\boxed{5cm}
\end{align*}
\item Posicion imagen $(w)$:
Distancia desde el vertice de la superficie hasta la imagen:
\begin{align*}
    \boxed{w=-5cm}
\end{align*}
\end{itemize}

%%%%%%%%%%%%%%%%%%%%%%%%%%%%%%%%%%%%%%%%%%%%



\clearpage
%%%%%%%%%%%%%%%%%%%%%%%%%%%%%%%%%%%%%%%%%%%%%%%%%%%%%%%%%%%%%%%%%%%%%%%%%%%%%%%%%%%%%%%%%%%%%%%%%%%%%%%%%%%%%%%%%%%%%%%%%%%%%%%%%%%%%%%%%%%%%%%%%%%%%%%%%%%%%%%%%%%%%%%%%%%%%%%%%%%%%%%%%%%%%%%%%%%%%%%

\section*{Situación 2}
Una lente delgada convergente de longitud focal $|f| = 15\,\text{cm}$ forma la imagen de un objeto colocado a una distancia de $5\,\text{cm}$ a la izquierda de la lente. 

\textbf{Solución:}

%%%%%%%%%%%%%%%%%%%%%%%%%%%%%%%%%%%%%%%%%%%%%%%%%%%%%%%%%%%%%%%%%%%%%%%%%%%%%%%%%%%%%%%%%%%%%%%%%%%%%%%%%%%%%%%%%%%%%%%%%%%%%%%%%%%%%%%%%%%%%%%%%%%%%%%%%%%%%%%%%%%%%%%%%%%%%%%%%%%%%%%%%%%%%%%%%%%%%%%

En este caso la matriz del sistema óptico consiste de una refracción en la \textbf{lente delgada convergente}. Con $f = +15\,\text{cm}$, se tiene:

\begin{equation}
    M =
    \left[\begin{matrix}
        1 & 0\\[2pt]
        -\frac{1}{f} & 1
    \end{matrix}\right]  =\boxed{\left[\begin{matrix}1 & 0\\[4pt]-\frac{1}{15}\,\text{cm}^{-1} & 1\end{matrix}\right]}\, ,
\end{equation}

donde $A=1$, $B=0$, $C=-\frac{1}{15}\,\text{cm}^{-1}$ y $D=1$.

Notar que el \textbf{determinante} de esta matriz es 
\begin{align*}
    \det M &= (1\cdot1)-\left[0\cdot\left(-\frac{1}{15}\,\text{cm}^{-1}\right)\right] \\
    &=1 \,,
\end{align*}
es decir, 
\begin{equation*}
    \boxed{\det M =1}
\end{equation*}
lo que implica
\begin{equation*}
    \frac{n_1}{n_3}=1 \quad \Rightarrow \quad n_1=n_3 \,,
\end{equation*}
es decir, la lente delgada esta inmersa en un mismo medio de indice de refracción $n_1$.

Ahora, calculamos los \textbf{puntos cardinales}:

\begin{align*}
    f =\left(\frac{n_1}{n_3}\right)\frac{1}{C}=(1)\left(\frac{1}{-\frac{1}{15}\,\text{cm}^{-1}}\right) \quad &\rightarrow \quad \boxed{f=-15 \, \text{cm}}\,. \\
    f' = -\frac{1}{C}=-\left(\frac{1}{-\frac{1}{15}\,\text{cm}^{-1}}\right) \quad &\rightarrow \quad \boxed{f'=+15 \, \text{cm}}\,. \\
    r = \left( D -\frac{n_1}{n_3}\right) \frac{1}{C} = (1-1)\left(\frac{1}{-\frac{1}{15}\,\text{cm}^{-1}}\right)\quad &\rightarrow \quad \boxed{r=0 \, \text{cm}}\,.\\
    s = (1 - A) \frac{1}{C}=(1-1)\left(\frac{1}{-\frac{1}{15}\,\text{cm}^{-1}}\right) \quad &\rightarrow \quad \boxed{s= 0\, \text{cm}}\,.\\
v = (D - 1)\frac{1}{C}\quad=(1-1)\left(\frac{1}{-\frac{1}{15}\,\text{cm}^{-1}}\right) &\rightarrow \quad \boxed{v= 0\, \text{cm}}\,.\\
w = \left(\frac{n_1}{n_3}- A\right)\frac{1}{C}=(1-1)\left(\frac{1}{-\frac{1}{15}\,\text{cm}^{-1}}\right) \quad &\rightarrow \quad \boxed{s= 0 \, \text{cm}}\,.
\end{align*}

En resumen:

\begin{itemize}
    \item El \textbf{primer foco} está a $15~\text{cm}$ a la izquierda del centro óptico de la lente.
    \item El \textbf{segundo foco} está a $15~\text{cm}$ a la derecha del centro óptico de la lente.
    \item El \textbf{centro óptico} de la lente se toma como el origen, es decir, $0~\text{cm}$.
    \item El \textbf{primer plano principal}, el \textbf{segundo plano principal}, el \textbf{primer plano nodal} y el \textbf{segundo plano nodal} coinciden en el centro de la lente para una lente delgada ideal, por lo tanto, se encuentran en $0~\text{cm}$.
\end{itemize}

Esto se resume en la siguiente tabla:

\begin{table}[H]
    \centering
    \begin{tabular}{|c|c|c|c|c|c|}\hline
         $f$&  $f'$&  $r$&  $s$&  $v$& $w$\\\hline
         $-15 \, \text{cm}$&  $+15 \, \text{cm}$&  $0\, \text{cm}$& $ 0\, \text{cm}$& $ 0\, \text{cm}$& $0\, \text{cm}$\\ \hline
    \end{tabular}
    \caption{Tabla con las distancias cardinales.}
\end{table}

La distancia imagen $z_2$ se obtiene a partir de los elementos de la matriz [\textbf{ABCD}] y la distancia objeto $z_1$. Con $z_1=5\,\text{cm}$ se tiene:
\begin{align*}
    z_2&=-\left( \frac{z_1A+B}{z_1C+D} \right) \\
    &=-\left( \frac{(5\,\text{cm})1+0}{(5\,\text{cm})\left(-\frac{1}{15}\,\text{cm}^{-1}\right)+1} \right) \\
    &=-\left( \frac{5\,\text{cm}}{\left(-\frac{1}{3}\right)+1} \right) \\
    &=-\left( \frac{5\,\text{cm}}{\frac{2}{3}} \right) \\[2pt]
    &= -\frac{15}{2} \, \text{cm} \, .
\end{align*}

Por lo tanto la distancia imagen $z_2$ es:

\begin{equation*}
    \boxed{z_2=-\frac{15}{2}  \, \text{cm} }\, .
\end{equation*}

Sabemos que para una matriz de planos conjugado el elemento A de esa matriz representa el aumento lateral del sistema, así:
\begin{align*}
   m &= A+z_2C\\
   &=1+(-\frac{15}{2}  \, \text{cm})\left(-\frac{1}{15}\,\text{cm}^{-1}\right)\\
   &=1+\frac{1}{2} \\
   &=\frac{3}{2}\ \, .
\end{align*}

Por lo tanto el aumento lateral del sistema es:

\begin{equation*}
    \boxed{m=\frac{3}{2} }\, .
\end{equation*}

\subsection*{Características de la imagen}

\begin{itemize}
  \item \textbf{Tipo:} virtual (la imagen aparece en el mismo lado que el objeto, $z_2=-\tfrac{15}{2}\,\mathrm{cm}<0$).
  \item \textbf{Orientación:} derecha (erecta, $m=+\tfrac{3}{2}>0$).
  \item \textbf{Tamaño:} aumentada, $|m| = \tfrac{3}{2}$ veces el objeto.
  \item \textbf{Escala:} en el eje $x$, cada cuadrito vale 1\,cm; en el eje $y$, cada 2 cuadritos valen 1\,cm.
\end{itemize}

\begin{figure}[H]
  \centering
  \includegraphics[width=0.7\textwidth]{sit_2.png}
  \caption{
    Construcción geométrica de la formación de imagen para la \textbf{Situación 2}: una lente delgada convergente de distancia focal $f = +15$\,cm, con un objeto situado a $z_1 = -5$\,cm a la izquierda de la lente (origen en el centro óptico $H_1$). Se muestran los puntos focales ($F_1$, $F_2$), los planos principales y nodales ($PH_1$, $PH_2$, $PO$), y la trayectoria de los rayos principales: \textbf{Rayo 1} (rojo) paralelo al eje óptico que tras la lente pasa por el foco posterior $F_2$; \textbf{Rayo 2} (verde) que atraviesa el foco anterior $F_1$ y emerge paralelo al eje óptico; \textbf{Rayo 3} (azul) que atraviesa el centro óptico sin desviarse. El cruce de las prolongaciones de los rayos determina la posición de la \textbf{imagen virtual}, situada a $z_2 = -7.5$\,cm, mayor y derecha respecto al objeto. Se indican además las distancias $z_1$, $z_2$, $f$ y $f'$ para referencia.
  }
  \label{fig:rayos-sit2}
\end{figure}


\textbf{Caracterización de los rayos principales:}

\begin{itemize}
  \item \textbf{Rayo 1 (R1):} Parte del objeto paralelo al eje óptico, atraviesa la lente y se refracta pasando por el foco posterior \(F_2\). En el caso de imagen virtual, su prolongación (línea punteada hacia la izquierda) converge hacia la posición de la imagen detrás del objeto.
  \item \textbf{Rayo 2 (R2):} Se dirige hacia el foco anterior \(F\) de la lente, y al atravesarla, emerge paralelo al eje óptico. Para la imagen virtual, la prolongación de este rayo también converge con la prolongación de R1 en la ubicación de la imagen.
  \item \textbf{Rayo 3 (R3):} Cruza el centro óptico de la lente y continúa en línea recta sin desviación, ya que en la aproximación de lente delgada el centro óptico no introduce desplazamiento angular.
  \item La \textbf{imagen virtual} se forma en el punto donde se cruzan las prolongaciones de $R1$ y $R2$. En este caso, la imagen aparece a la izquierda de la lente, es mayor que el objeto y está derecha, coherente con el aumento positivo.
\end{itemize}

\clearpage
%%%%%%%%%%%%%%%%%%%%%%%%%%%%%%%%%%%%%%%%%%%%%%%%%%%%%%%%%%%%%%%%%%%%%%%%%%%%%%%%%%%%%%%%%%%%%%%%%%%%%%%%%%%%%%%%%%%%%%%%%%%%%%%%%%%%%%%%%%%%%%%%%%%%%%%%%%%%%%%%%%%%%%%%%%%%%%%%%%%%%%%%%%%%%%%%%%%%%%%

\section*{Situación 3}
A la derecha de la lente de la Situación 2, y a una distancia de $60\,\text{cm}$ desde su centro, se coloca una segunda lente delgada divergente de longitud focal $|f| = 15\,\text{cm}$. Un objeto es colocado a $25\,\text{cm}$ a la izquierda de la primera lente.

\textbf{Solución:}

%%%%%%%%%%%%%%%%%%%%%%%%%%%%%%%%%%%%%%%%%%%%%%%%%%%%%%%%%%%%%%%%%%%%%%%%%%%%%%%%%%%%%%%%%%%%%%%%%%%%%%%%%%%%%%%%%%%%%%%%%%%%%%%%%%%%%%%%%%%%%%%%%%%%%%%%%%%%%%%%%%%%%%%%%%%%%%%%%%%%%%%%%%%%%%%%%%%%%%%

En este caso, la matriz del sistema óptico consiste en una refracción en la primera lente delgada convergente, una propagación entre las lentes y una última refracción en la segunda lente delgada divergente. Con $f = +15\,\text{cm}$, $d = 60\,\text{cm}$ y $f' = -15\,\text{cm}$, se tiene:

\begin{align*}
    M &=
    \left[\begin{matrix}
        1 & 0\\[2pt]
        -\frac{1}{f'} & 1
    \end{matrix}\right]  
    \left[\begin{matrix}
        1 & d \\
        0 & 1 
    \end{matrix}\right] \left[\begin{matrix}
        1 & 0\\[2pt]
        -\frac{1}{f} & 1
    \end{matrix}\right] \\[3pt]
    &=
    \left[\begin{matrix}
        1 & 0\\[2pt]
        -\frac{1}{-15\,\text{cm}} & 1
    \end{matrix}\right]  
    \left[\begin{matrix}
        1 & 60\,\text{cm} \\
        0 & 1 
    \end{matrix}\right] \left[\begin{matrix}
        1 & 0\\[2pt]
        -\frac{1}{15\,\text{cm}} & 1
    \end{matrix}\right] \\[3pt]
    &=\left[\begin{matrix}
        1 & 0\\[2pt]
        \frac{1}{15}\,\text{cm}^{-1} & 1
    \end{matrix}\right]  
    \left[\begin{matrix}
        -3 & 60\,\text{cm} \\
        -\frac{1}{15} \,\text{cm}^{-1}& 1 
    \end{matrix}\right] \\
    &=\left[\begin{matrix}
        -3 & 60 \,\text{cm}\\[2pt]
        -\frac{3}{15}\,\text{cm}^{-1}-\frac{1}{15}\,\text{cm}^{-1} & 5
    \end{matrix}\right]  \\
    &=\left[\begin{matrix}
        -3 & 60 \,\text{cm}\\[2pt]
         -\frac{4}{15}\,\text{cm}^{-1} & 5
    \end{matrix}\right]  \,.
\end{align*}

Por lo tanto, la matriz de trayectoria de un rayo o matriz [\textbf{ABCD}] que caracteriza el sistema es:
\begin{equation*}
    \boxed{M=\left[\begin{matrix}
        -3 & 60 \,\text{cm}\\[2pt]
         -\frac{4}{15}\,\text{cm}^{-1} & 5
    \end{matrix}\right]  } \,,
\end{equation*}

donde $A=-3$, $B=60 \,\text{cm}$, $C=-\frac{4}{15}\,\text{cm}^{-1}$ y $D=5$.

Notar que el \textbf{determinante} de esta matriz es 
\begin{align*}
    \det M &= (-3)\cdot(5)-(60 \,\text{cm})\cdot\left(-\frac{4}{15}\,\text{cm}^{-1}\right) \\
    &=-15+16\\
    &=1
\end{align*}
es decir, 
\begin{equation*}
    \boxed{\det M =1}
\end{equation*}
lo que implica
\begin{equation*}
    \frac{n_1}{n_3}=1 \quad \Rightarrow \quad n_1=n_3 \,,
\end{equation*}
es decir, el sistema óptico esta inmerso en un mismo medio de indice de refracción $n_1$.

Ahora, calculamos los \textbf{puntos cardinales}:

\begin{align*}
    f =\left(\frac{n_1}{n_3}\right)\frac{1}{C}=(1)\left(\frac{1}{-\frac{4}{15}\,\text{cm}^{-1}}\right) \quad &\rightarrow \quad \boxed{f=-\frac{15}{4} \, \text{cm}}\,. \\
    f' = -\frac{1}{C}=-\left(\frac{1}{-\frac{4}{15}\,\text{cm}^{-1}}\right) \quad &\rightarrow \quad \boxed{f'=+\frac{15}{4} \, \text{cm}}\,. \\
    r = \left( D -\frac{n_1}{n_3}\right) \frac{1}{C} = (5-1)\left(\frac{1}{-\frac{4}{15}\,\text{cm}^{-1}}\right)\quad &\rightarrow \quad \boxed{r=-15 \, \text{cm}}\,.\\
    s = (1 - A) \frac{1}{C}=[1-(-3)]\left(\frac{1}{-\frac{4}{15}\,\text{cm}^{-1}}\right) \quad &\rightarrow \quad \boxed{s= -15\, \text{cm}}\,.\\
v = (D - 1)\frac{1}{C}\quad=(5-1)\left(\frac{1}{-\frac{4}{15}\,\text{cm}^{-1}}\right) &\rightarrow \quad \boxed{v= -15\, \text{cm}}\,.\\
w = \left(\frac{n_1}{n_3}- A\right)\frac{1}{C}=[1-(-3)]\left(\frac{1}{-\frac{4}{15}\,\text{cm}^{-1}}\right) \quad &\rightarrow \quad \boxed{s= -15 \, \text{cm}}\,.
\end{align*}

En resumen:

\begin{itemize}
    \item El \textbf{primer foco} está a $\dfrac{15}{4}~\text{cm}$ a la izquierda del centro óptico de la lente.
    \item El \textbf{segundo foco} está a $\dfrac{15}{4}~\text{cm}$ a la derecha del centro óptico de la lente.
    \item El \textbf{centro óptico} de la lente se ubica en $-15~\text{cm}$.
    \item El \textbf{primer plano principal}, el \textbf{segundo plano principal}, el \textbf{primer plano nodal} y el \textbf{segundo plano nodal} coinciden en el centro óptico de la lente para una lente delgada ideal, por lo tanto, se encuentran todos en $-15~\text{cm}$.
\end{itemize}

Los puntos cardinales del sistema compuesto de dos lentes, obtenidos a partir de la matriz total del sistema (con origen de coordenadas en el centro óptico de la primera lente, $z=0$), son:

Esto se resume en la siguiente tabla:

\begin{table}[h]
    \centering
    \begin{tabular}{|c|c|c|c|c|c|}\hline
         $f$&  $f'$&  $r$&  $s$&  $v$& $w$\\\hline
         $-\frac{15}{4} \, \text{cm}$&  $+\frac{15}{4} \, \text{cm}$&  $-15\, \text{cm}$&  $-15\, \text{cm}$&  $-15\, \text{cm}$& $-15\, \text{cm}$\\ \hline
    \end{tabular}
    \caption{Tabla con las distancias cardinales.}
\end{table}

La distancia imagen $z_2$ se obtiene a partir de los elementos de la matriz [\textbf{ABCD}] y la distancia objeto $z_1$. Con $z_1=25\,\text{cm}$ se tiene:
\begin{align*}
    z_2&=-\left( \frac{z_1A+B}{z_1C+D} \right) \\
    &=-\left( \frac{(25\,\text{cm})(-3)+(60 \,\text{cm})}{(25\,\text{cm})\left(-\frac{4}{15}\,\text{cm}^{-1}\right)+5} \right) \\
    &=-\left( \frac{-75\,\text{cm}+60\,\text{cm}}{-\frac{20}{3}+5} \right) \\
    &=-\left( \frac{-15\,\text{cm}}{\frac{-20+15}{3}} \right) \\[2pt]
    &= -\frac{15}{\frac{5}{3}} \, \text{cm} \\
    &= -9 \, \text{cm} 
\end{align*}

Por lo tanto la distancia imagen $z_2$ es:

\begin{equation*}
    \boxed{z_2=-9  \, \text{cm} }\, .
\end{equation*}

Sabemos que para una matriz de planos conjugado el elemento A de esa matriz representa el aumento lateral del sistema, así:
\begin{align*}
   m &= A+z_2C\\
   &=-3+(-9\,\text{cm})\left(-\frac{4}{15}\,\text{cm}^{-1}\right)\\
   &=-3+\frac{12}{5} \\
   &=-\frac{3}{5} \, .
\end{align*}

Por lo tanto el aumento lateral del sistema es:

\begin{equation*}
    \boxed{m=-\frac{3}{5}  }\, .
\end{equation*}

La imagen se ubica a $z_2 = -9$~cm, es decir, a la izquierda de la primera lente, lo cual indica que es una imagen virtual. El valor negativo del aumento ($m=-\tfrac{3}{5}$) indica que la imagen está invertida respecto al objeto, y el valor absoluto menor que uno implica que es reducida.

\subsection{Características de la imagen}
 
\begin{itemize}
  \item \textbf{Tipo:} virtual (la imagen aparece en el mismo lado que el objeto, $z_2=-9\;\mathrm{cm}<0$).
  \item \textbf{Orientación:} invertida (debido a que $m=-\tfrac{3}{5}<0$).
  \item \textbf{Tamaño:} reducida, $|m| = \tfrac{3}{5}$ veces el objeto.
  \item \textbf{Escala:} en el eje $x$, cada cuadrito vale 1\,cm; en el eje $y$, cada 2 cuadritos valen 1\,cm.
\end{itemize}
 
\begin{figure}[H]
  \centering
  \includegraphics[width=0.8\textwidth]{sit_3.png}
  \caption{Diagrama de rayos (Situación 3).  
    Objeto a $z_1 = -25\,$cm, lente 1 en $z=0$, lente 2 en $z=+60\,$cm, imagen virtual en $z_2=-9\,$cm (a 9\,cm a la izquierda de la lente 2).}
  \label{fig:rayos-sit3}
\end{figure}
 
\vspace{1em}
\noindent
\textbf{Dibujo del diagrama de rayos:}
\begin{itemize}
  \item \textbf{R1} – Rayo paralelo al eje óptico que, tras la lente convergente, se refracta pasando por su foco posterior, se propaga 60\,cm y luego, al atravesar la lente divergente, sale como si proviniera de su foco anterior.  
  \item \textbf{R2} – Rayo dirigido hacia el foco anterior de la primera lente que, tras ella, emerge paralelo; después de los 60\,cm atraviesa la segunda lente y se desvía como rayo divergente paralelo al eje (prolongación apunta hacia la izquierda).  
  \item \textbf{R3} – Rayo que atraviesa el centro óptico de la primera lente sin desviarse, viaja 60\,cm y atraviesa el centro óptico de la segunda lente también sin desviarse.  
  \item El cruce de las prolongaciones de R1 y R2 (líneas punteadas) marca la posición de la \emph{imagen virtual} entre las dos lentes.
\end{itemize}

\clearpage
%%%%%%%%%%%%%%%%%%%%%%%%%%%%%%%%%%%%%%%%%%%%%%%%%%%%%%%%%%%%%%%%%%%%%%%%%%%%%%%%%%%%%%%%%%%%%%%%%%%%%%%%%%%%%%%%%%%%%%%%%%%%%%%%%%%%%%%%%%%%%%%%%%%%%%%%%%%%%%%%%%%%%%%%%%%%%%%%%%%%%%%%%%%%%%%%%%%%%%%

\section{Problema 1}

    \textbf{A.} Demuestre, analíticamente, que una placa SELFOC de longitud $d <\pi/2\alpha$ e índice de refracción dado por la expresión
    \begin{equation}
        n^2(y) = n_0^2 \left( 1 - \alpha^2 y^2 \right)
    \end{equation}
    actúa como una lente cilíndrica (una lente con poder focalizador en el plano $y-z$) de longitud focal
    \begin{equation}
        f_2 = \frac{1}{\alpha \sin (\alpha d)}\, .
    \end{equation}
    \begin{figure}[H]
        \centering
        \includegraphics[width=0.5\linewidth]{fig1.png}
        \caption{La placa SELFOC usada como una lente; $f_2$ es el segundo punto focal y $H_2$ es el segundo punto principal.}
        \label{fig:1_a_fig}
    \end{figure}
\textbf{Solución:} 
%%%%%%%%%%%%%%%%%%%%%%%%%%%%%%%%%%%%%%%%%%%%%%%%%%%%%%%%%%%%%%%%%%%%%%%%%%%%%%%%%%%%%%%%%%%%%%%%%%%%%%%%%%%%%%%%%%%%%%%%%%%%%%%%%%%%%%%%%%%%%%%%%%%%%%%%%%%%%%%%%%%%%%%%%%%%%%%%%%%%%%%%%%%%%%%%%%%%%%%

Consideremos un rayo propagándose por una placa SELFOC por el plano $y-z$, con eje óptico paralelo al eje $z$. Dicha placa tiene indice de refracción homogéneo en $x$ y $z$, pero variable en $y$ de forma:
\begin{equation}\label{p1_coeficiente_refracción}
        n^2(y) = n_0^2 \left( 1 - \alpha^2 y^2 \right) \,,
\end{equation}

Considerando que $n(y)$ es una función siempre positiva y $n_0>0$, tomamos la raíz cuadrada de esta ecuación y la expresamos de forma:
\begin{equation}\label{p1_index}
    n(y)=n_0\,(1-\alpha^2\,y^2)^{\frac{1}{2}}
\end{equation}

Consideramos el desarrollo en serie de Taylor para una función $f(x) = (1 - x)^n$,

\begin{equation}\label{p1_taylor}
    (1 - x)^n = 1 - nx + \frac{n(n-1)}{2}x^2 - \frac{n(n-1)(n-2)}{3!}x^3 + \cdots 
\end{equation}

Tomando $x=\alpha^2\,y^2$ y $n=\frac{1}{2}$ en \eqref{p1_taylor}. A su vez, suponiendo que $\alpha$ es tal que $\alpha^2\,y^2 \ll1$, para todo $y$ de interés. Se obtiene la aproximación:
\begin{equation}\label{p1_taylor_approx}
    (1 - \alpha^2\,y^2)^{\frac{1}{2}} \approx 1 - \frac{1}{2}\alpha^2\,y^2
\end{equation}

Empleando la aproximación \eqref{p1_taylor_approx} en \eqref{p1_index}, tenemos:

\begin{equation*}
    n(y)\approx n_0\,\left(1 - \frac{1}{2}\alpha^2\,y^2\right)
\end{equation*}

Reordenamos factores

\begin{equation*}
\frac{n_0}{n(y)} \approx \left(1 - \frac{1}{2}\alpha^2 y^2\right)^{-1}\,.
\end{equation*}
Utilizando el desarrollo en serie de Taylor para una función $f(x) = (1 - x)^{-1}$:
\begin{equation*}
(1-x)^{-1} = 1 + x + x^2 + x^3 + \cdots\,,\qquad |x| < 1\,,
\end{equation*}
teniendo $x = \frac{1}{2}\alpha^2 y^2$, obtenemos:
\begin{equation*}
\frac{n_0}{n(y)} \approx 1 + \frac{1}{2}\alpha^2 y^2 + \left(\frac{1}{2}\alpha^2 y^2\right)^2 + \cdots\,.
\end{equation*}

Dado que $\alpha^2 y^2 \ll 1$, el término de orden $\alpha^4 y^4$ puede despreciarse,
\begin{equation*}
\frac{n_0}{n(y)} \approx 1 + \frac{1}{2}\alpha^2 y^2 \,,
\end{equation*}
Elevamos esta aproximación al cuadrado, tenemos:
\begin{align*}
\left(\frac{n_0}{n(y)}\right)^2 &\approx \left(1 + \frac{1}{2}\alpha^2 y^2\right)^2 \\
&= 1 + 2 \left(\frac{1}{2}\alpha^2 y^2\right) + \left(\frac{1}{2}\alpha^2 y^2\right)^2 \\
&= 1 + \alpha^2 y^2 + \frac{1}{4}\alpha^4 y^4\,.
\end{align*}

Nuevamente, dado que $\alpha^2 y^2 \ll 1$, el término de orden $\alpha^4 y^4$ puede despreciarse. Obtenemos así:
\begin{equation}\label{p1_aproximación_n_0/n(y)}
\left(\frac{n_0}{n(y)}\right)^2 \approx 1 + \alpha^2 y^2\,.
\end{equation}

A continuación, derivamos la ecuación \eqref{p1_coeficiente_refracción} respecto a $y$:

\begin{equation*}
    \frac{d}{dy}[n^2(y)]=\frac{d}{dy}\left[n_0^2\,(1-\alpha^2\,y^2)\right] \, ,
\end{equation*}

obtenemos:

\begin{equation}\label{p1_dn/dy}
    \frac{dn}{dy}(y)=-\frac{n_0^2}{n(y)}\,\alpha^2y \,.
\end{equation}

Sabemos que la ecuación de un rayo en un medio con índice variable es, en general,
\begin{equation}
    \frac{d}{ds}\left(n\, \frac{d\vec{r}}{ds}\right) = \nabla n\,,
\end{equation}
donde $s$ es la longitud de arco, y $\vec{r}(s)$ la posición del rayo. Para rayos que se propagan en el plano $y$-$z$ (eje óptico en $z$) y variación sólo en $y$, esto se reduce a
\begin{equation}
    \frac{d}{ds}\left(n\, \frac{dy}{ds}\right) = \frac{d n}{d y}\,.
    \label{eq:trayectoria_general}
\end{equation}

En la aproximación paraxial ($\theta \approx dy/dz \ll 1$, $ds \approx dz$), esto se convierte en:
\begin{equation}
    \frac{d}{dz}\left(n\, \frac{dy}{dz}\right) \approx \frac{dn}{d y}\,.
    \label{eq:ecuacion_paraxial}
\end{equation}

Dado que el indicie de refracción $n$ es no homogéneo solo en $y$, es decir $n=n(y)$, se tiene:

\begin{equation*}
    n(y)\frac{d^2y}{dz^2} =\frac{d n}{d y} \,.
\end{equation*}

Dividiendo ambos lados por $n(y)$ (asumiendo $n(y) \neq 0$ en la región relevante),
\begin{equation}\label{p1_edo}
    \frac{d^2y}{dz^2} = \frac{1}{n(y)}\frac{d n}{d y}.
\end{equation}

Sustituyendo \eqref{p1_dn/dy} en \eqref{p1_edo},

\begin{equation} \label{p1_edo_2}
    \frac{d^2y}{dz^2} \approx -\left(\frac{n_0}{n(y)}\right)^2\,\alpha^2y.
\end{equation}

Sustituyendo \eqref{p1_aproximación_n_0/n(y)} en \eqref{p1_edo_2}, 

\begin{equation*} 
    \frac{d^2y}{dz^2} \approx -\left(1 + \alpha^2 y^2\right)\,\alpha^2y=-\left(\alpha^2y + \alpha^4 y^3\right)\,.
\end{equation*}

Siendo $\alpha^2 y^2 \ll 1$, el término $\alpha^4 y^3$ puede despreciarse a primer orden, quedando

\begin{equation*}
    \frac{d^2y}{dz^2} \approx -\,\alpha^2y \,.
\end{equation*}

Así, la ecuación para la trayectoria del rayo a travez de la placa SELFOC es:

\begin{equation}
    \frac{d^2 y}{dz^2} + \alpha^2 y = 0.
    \label{eq:edo_oscilador}
\end{equation}

Esto es la ecuación de un \textbf{oscilador armónico simple} de frecuencia $\alpha$. Cuya solución general de es:
\begin{equation*}
    y(z) = A \cos(\alpha z) + B \sin(\alpha z) \,.
\end{equation*}
Supongamos una posición inicial $y(0) = y_0$ y una pendiente inicial $\frac{dy}{dz}= \theta_0$ en $z=0$. Con tales condiciones iniciales se definen los coeficientes $A$ y $B$, de modo:
\begin{equation*}
    A=y_0 \quad  \text{y} \quad B=\frac{\theta_0}{\alpha} \,.
\end{equation*}

Así, se tiene la siguiente ecuación para la trayectoria del rayo:
\begin{equation*}
    y(z) = y_0 \cos (\alpha z) + \frac{\theta_0}{\alpha} \sin (\alpha z) \,.
\end{equation*}
De la ecuación anterior se tiene que la pendiente de la trayectoria es
\begin{equation*}
    \frac{dy}{dz}(z) = -y_0 \,\alpha \sin (\alpha z) + \theta_0 \cos (\alpha z) \,.
\end{equation*}


Al recorrer la placa SELFOC de longitud $d$, el estado del rayo evoluciona de $(y_0, \theta_0)$ en $z=0$ a $(y_d, \theta_d)$ en $z=d$:
\begin{align*}
    y(d) &= y_0 \cos(\alpha d) + \frac{\theta_0}{\alpha} \sin(\alpha d) \\[1ex]
    \frac{dy}{dz}(d) &= -y_0 \alpha \sin(\alpha d) + \theta_0 \cos(\alpha d)
\end{align*}

Es decir, la evolución es lineal y se puede escribir en forma matricial:
\begin{equation}
    \begin{pmatrix}
        y(d) \\[2pt]
        \theta(d)
    \end{pmatrix}
    =
    \begin{pmatrix}
        \cos(\alpha d) & \dfrac{1}{\alpha}\sin(\alpha d) \\[8pt]
        -\alpha\sin(\alpha d) & \cos(\alpha d)
    \end{pmatrix}
    \begin{pmatrix}
        y_0 \\[2pt]
        \theta_0
    \end{pmatrix}
    \label{eq:matriz_selfoc}
\end{equation}

Identificamos que en este sistema optico la \textbf{matriz de transformación del rayo} es

\begin{equation*}
    M=\begin{pmatrix}
    A & B \\ C & D
\end{pmatrix} = \begin{pmatrix}
        \cos(\alpha d) & \dfrac{1}{\alpha}\sin(\alpha d) \\[8pt]
        -\alpha\sin(\alpha d) & \cos(\alpha d)
    \end{pmatrix}
\end{equation*}

Queremos demostrar que este sistema actúa como una \textbf{lente cilíndrica} y encontrar su distancia focal de salida $f_2$. Es sabido que $f_2$ puede calcularse a partir de $C$ de modo:
\begin{equation}    \label{eq:f2_de_C}
    f_2 = \frac{1}{-C} \, ,
\end{equation}

donde $C=-\alpha\sin(\alpha d)$. Sustituyendo en \eqref{eq:f2_de_C}, obtenemos:
\begin{equation*}
    f_2 = \frac{1}{\alpha \sin(\alpha d)}    
\end{equation*}

Hemos demostrado que una placa SELFOC de perfil cuadrático, longitud $d$, y parámetro $\alpha$ actúa, en régimen paraxial, como una lente cilíndrica cuya distancia focal de salida es
\begin{equation*}
\boxed{
f_2 = \frac{1}{\alpha \sin(\alpha d)}
}    
\end{equation*}

válida mientras $d < \frac{\pi}{2\alpha}$, pues para mayores valores la función seno se anula y la trayectoria de los rayos se vuelve periódica (no focalizadora).


\clearpage
%%%%%%%%%%%%%%%%%%%%%%%%%%%%%%%%%%%%%%%%%%%%%%%%%%%%%%%%%%%%%%%%%%%%%%%%%%%%%%%%%%%%%%%%%%%%%%%%%%%%%%%%%%%%%%%%%%%%%%%%%%%%%%%%%%%%%%%%%%%%%%%%%%%%%%%%%%%%%%%%%%%%%%%%%%%%%%%%%%%%%%%%%%%%%%%%%%%%%%%
    \textbf{B.} Demuestre que el segundo plano principal $H_2$ (mostrado en la figura adjunta) reside a la distancia
    \begin{equation*}
        s = - \left(\frac{1}{\alpha}\right) \tan \left(\frac{\alpha d}{2} \right)\,.
    \end{equation*}
    
\textbf{Solución:} 
%%%%%%%%%%%%%%%%%%%%%%%%%%%%%%%%%%%%%%%%%%%%%%%%%%%%%%%%%%%%%%%%%%%%%%%%%%%%%%%%%%%%%%%%%%%%%%%%%%%%%%%%%%%%%%%%%%%%%%%%%%%%%%%%%%%%%%%%%%%%%%%%%%%%%%%%%%%%%%%%%%%%%%%%%%%%%%%%%%%%%%%%%%%%%%%%%%%%%%%

Para el calculo de $s$, consideremos nuevamente la matriz de transferencia del rayo de este sistema óptico formado por la placa SELFOC,
\begin{equation*}
    M=\begin{pmatrix}
    A & B \\ C & D
\end{pmatrix} = \begin{pmatrix}
        \cos(\alpha d) & \dfrac{1}{\alpha}\sin(\alpha d) \\[8pt]
        -\alpha\sin(\alpha d) & \cos(\alpha d)
    \end{pmatrix} \,.
\end{equation*}

Sabemos que la distancia $s$ si relaciona con los elementos $A$ y $C$ de esta matriz según la ecuación:
\begin{equation}\label{p1_s}
    s=(1-A)\frac{1}{C} \,.
\end{equation}

Donde tenemos que:
\begin{equation*}
    A=\cos(\alpha d) \quad \text{y} \quad C=-\alpha\sin(\alpha d) \,.
\end{equation*}
Sustituyendo $A$ y $C$ en \eqref{p1_s}, tenemos:
\begin{equation*}
     s=[1-\cos(\alpha d)]\left(\frac{1}{-\alpha\sin(\alpha d)}\right) \,.
\end{equation*}
Expresamos la ecuación de forma conveniente:
\begin{equation}\label{1b_sss}
     s =-\left(\frac{1}{\alpha}\right)\left(\frac{1-\cos(\alpha d)}{\sin(\alpha d)}\right) \,.
\end{equation}
Recordemos las identidades trigonométricas:
\begin{equation*}
    \sin^2\left(\frac{\theta}{2}\right)=\frac{1}{2}-\frac{\cos\theta}{2} \,.
\end{equation*}
\begin{equation*}
    \sin(2\varphi) =2\sin(\varphi)\cos(\varphi) \,.
\end{equation*}

Definiendo de manera conveniente a los ángulos $\theta:=\alpha d$ y $\varphi:=\frac{\alpha d}{2}$, estas identidades se expresan de forma:

\begin{equation}\label{p1_b_sin^2}
    2\sin^2\left(\frac{\alpha d}{2}\right)=1-\cos(\alpha d) \,.
\end{equation}
\begin{equation}\label{p1_b_sin}
    \sin\left(\alpha d\right) =2\sin\left(\frac{\alpha d}{2}\right)\cos\left(\frac{\alpha d}{2}\right) \,.
\end{equation}

Sustituyendo \eqref{p1_b_sin^2} y \eqref{p1_b_sin} en \eqref{1b_sss}, se tiene:

\begin{align*}
    s &=-\left(\frac{1}{\alpha}\right)\left(\frac{1-\cos(\alpha d)}{\sin(\alpha d)}\right) \\[2pt]
    &=-\left(\frac{1}{\alpha}\right)\left(\frac{2\sin^2\left(\frac{\alpha d}{2}\right)}{2\sin\left(\frac{\alpha d}{2}\right)\cos\left(\frac{\alpha d}{2}\right)}\right) \\[2pt]
    &=-\left(\frac{1}{\alpha}\right)\left(\frac{\sin\left(\frac{\alpha d}{2}\right)}{\cos\left(\frac{\alpha d}{2}\right)}\right)\\[2pt]
    &=-\left(\frac{1}{\alpha}\right)\tan\left(\frac{\alpha d}{2}\right)
\end{align*}
    
Así, queda demostrado que el segundo plano principal $H_2$ (mostrado en la figura adjunta) reside a la distancia
    \begin{equation*}
        \boxed{s = - \left(\frac{1}{\alpha}\right) \tan \left(\frac{\alpha d}{2} \right)}\,.
    \end{equation*}
    
\clearpage
%%%%%%%%%%%%%%%%%%%%%%%%%%%%%%%%%%%%%%%%%%%%%%%%%%%%%%%%%%%%%%%%%%%%%%%%%%%%%%%%%%%%%%%%%%%%%%%%%%%%%%%%%%%%%%%%%%%%%%%%%%%%%%%%%%%%%%%%%%%%%%%%%%%%%%%%%%%%%%%%%%%%%%%%%%%%%%%%%%%%%%%%%%%%%%%%%%%%%%%
    \textbf{C.} Haga gráficos (separados) para mostrar la trayectoria del rayo mostrado en la Fig.~\ref{fig:1_a_fig} en los siguientes casos particulares:
    \begin{enumerate}
        \item[i)] $d = \frac{\pi}{\alpha}$
        \item[ii)] $d = \frac{\pi}{2\alpha}$
        \item[iii)] $d = \frac{\pi}{4\alpha}$
    \end{enumerate}
    Para sus gráficos considere $\alpha = 0.5$ e $y_0 = 2.5$. Para cada caso evalúe $s$ y $f$.


\textbf{Solución:} 
%%%%%%%%%%%%%%%%%%%%%%%%%%%%%%%%%%%%%%%%%%%%%%%%%%%%%%%%%%%%%%%%%%%%%%%%%%%%%%%%%%%%%%%%%%%%%%%%%%%%%%%%%%%%%%%%%%%%%%%%%%%%%%%%%%%%%%%%%%%%%%%%%%%%%%%%%%%%%%%%%%%%%%%%%%%%%%%%%%%%%%%%%%%%%%%%%%%%%%%
\subsection{c)}
Para abordar este ejercicio desde la óptica matricial debemos notar ciertas cosas, como que el vector de rayos se representa como:
\begin{align*}
    \left[\begin{matrix}
        y\\\theta
    \end{matrix}\right]
\end{align*}
y se transforma a través del sistema óptico por:
\begin{align*}
\left[\begin{matrix}
    y_{ex}\\
    \theta_{ex}
\end{matrix}\right]=\left[\begin{matrix}
    A&B\\C&D
\end{matrix}\right]\left[\begin{matrix}
    y_{in}\\
    \theta_{in}
\end{matrix}\right]
\end{align*}
\subsubsection{Paso 1}
Primero identificamos el sistema optico, para esto es necesario tener en cuenta que el indice de refraccion variable se nos da por enunciado y es $n^2(y)=n^2_0(1-\alpha^2y^2)$. Notemos que este medio puede ser representado mediante una matriz de trasnferencia ABCD en coordenadas paraxiales. La matriz de propagación a través de una longitud $d$ es:
\begin{align*}
    M_{GRIN}=\left[\begin{matrix}
        \cos({\alpha d}) &\frac{1}{d}\sin({\alpha d})\\-\alpha\sin({\alpha d})&\cos({\alpha d})
    \end{matrix}\right]
\end{align*}
La salida del rayo se obtiene aplicando:
\begin{align*}
    \left[\begin{matrix}
        y_d\\\theta_d
    \end{matrix}\right]=\left[\begin{matrix}
        \cos({\alpha d}) &\frac{1}{d}\sin({\alpha d})\\-\alpha\sin({\alpha d})&\cos({\alpha d})
    \end{matrix}\right]\cdot\left[\begin{matrix}
        y_0\\\theta_0
    \end{matrix}\right]
\end{align*}
Notemos que $\theta=0$, con esto el resultado se simplifica a:
\begin{align*}
    y_d&=y_0\cos(\alpha d)\\
    \theta_d&=-\alpha y_0\sin(\alpha   d)
\end{align*}
\subsubsection{Paso 2}
Ahora veámoslo por casos

Caso 1: $d=\frac{\pi}{\alpha}$
\begin{itemize}
    \item $\alpha d=\pi$
    \item $\cos(\pi)=-1,\sin(\pi)=0$
    \item $y_d=-2.5,\theta_d=0$
\end{itemize}
\includegraphics[scale=0.7]{image0.png}\\
Caso 2: $d=\frac{\pi}{2\alpha}$
\begin{itemize}
    \item $\alpha d=\frac{\pi}{2}$
    \item $\cos(\frac{\pi}{2})=0,\sin(\frac{\pi}{2})=1$
    \item $y_d=0,\theta_d=-1.25$
    \end{itemize}
    \includegraphics[scale=0.7]{image.png}\\
Caso 3: $d=\frac{\pi}{4\alpha}$
\begin{itemize}
    \item $\alpha d=\frac{\pi}{4}$
    \item $\cos(\frac{\pi}{4})=\frac{\sqrt{2}}{2},\sin(\frac{\pi}{4})=\frac{\sqrt{2}}{2}$
    \item $y_d=2.5\cdot\frac{\sqrt{2}}{2}\approx1.768,\theta_d=-0.5\cdot 2.5\cdot\frac{\sqrt{2}}{2}\approx-0.884$
\end{itemize}
\includegraphics[scale=0.7]{image.png}

\clearpage
%%%%%%%%%%%%%%%%%%%%%%%%%%%%%%%%%%%%%%%%%%%%%%%%%%%%%%%%%%%%%%%%%%%%%%%%%%%%%%%%%%%%%%%%%%%%%%%%%%%%%%%%%%%%%%%%%%%%%%%%%%%%%%%%%%%%%%%%%%%%%%%%%%%%%%%%%%%%%%%%%%%%%%%%%%%%%%%%%%%%%%%%%%%%%%%%%%%%%%%
\section{Problema 2}

Con base en una lente bi-convexa genérica de radios de curvatura $R_1$ y $R_2$ y de espesor central $t$ derive las Ecs.~(1) a la (6) del archivo \textit{Óptica-Matricial-1.pdf}. La lente es de material óptico transparente de índice de refracción $n_\ell$ y a la izquierda limita con un material óptico transparente de índice de refracción $n_1$, y a la derecha, con uno de índice de refracción $n_2$.

\textbf{Solución:} 
%%%%%%%%%%%%%%%%%%%%%%%%%%%%%%%%%%%%%%%%%%%%%%%%%%%%%%%%%%%%%%%%%%%%%%%%%%%%%%%%%%%%%%%%%%%%%%%%%%%%%%%%%%%%%%%%%%%%%%%%%%%%%%%%%%%%%%%%%%%%%%%%%%%%%%%%%%%%%%%%%%%%%%%%%%%%%%%%%%%%%%%%%%%%%%%%%%%%%%%

En el problema se analiza un lente bi--convexa con radios \(R_1 > 0,   R_2 < 0\), un espesor
central \(t\), con índice interno \(n_\ell\) (material del lente) y
dos medios exteriores de índices \(n_1\) que seria del lado del objeto(medio entrante) y
\(n_2\) del lado de la imagen (medio saliente) . Esto con el objetivo de construir su matriz global \(M\) y, a partir de ella, obtener lo siguiente:

\begin{enumerate}
  \item Longitudes focales \(f_1\) y \(f_2\).
  \item Posición de los planos principales \(H_1,H_2\).
  \item Posición de los puntos nodales \(N_1,N_2\).
  \item Ecuación de Gauss gruesa y aumento \(m\).
  \item Límite de lente delgada en aire.
\end{enumerate}

La gracia en la óptica paraxial es que cada rayo se puede codificar mediante un vector
\[
\mathbf r=\begin{pmatrix}y\\ n\theta\end{pmatrix},
\]
donde \(y\) es la altura sobre el eje óptico y \(n\theta\) seria su
\emph{ángulo reducido}.

Para seguir con el desarrollo se usaran convenciones y aproximaciones básicas.Considerando el eje óptico $z$ con direccion hacia la derecha, y $y>0$ arriba y $\theta>0$ cuando el rayo se esta  inclina hacia arriba. $R$ se toma como positiva si es convexa hacia la izquierda y negativa si es convexa hacia la derecha. Con la aproximación paraxial, los ángulos son muy pequeños
\[
\tan\theta\approx\theta,\quad
\sin\theta\approx\theta,\quad
\cos\theta\approx1.
\]
Facilitando linealizar las leyes de Snell y la geometría de propagación, para que cada elemento del sistema óptico queda descrito por una transformación lineal en $(y,\theta)$.


\subsection{Matriz $M_{1}$: refracción en la primera superficie}

La refracción de un rayo que incide desde el medio $n_{1}$ a $n_{\ell}$ en una superficie de radio $R_{1}$, usamos la ley de Snell linealizada :  
\[
n_{1}\,\alpha_{1} = n_{\ell}\,\alpha'
\quad\Longrightarrow\quad
\alpha' = \frac{n_{1}}{n_{\ell}}\Bigl(\theta_{1} - \frac{y_{1}}{R_{1}}\Bigr).
\]
Se suma la inclinación de la normal, asi obteniendo el ángulo respecto al eje:  
\[
\theta' = \alpha' + \frac{y_{1}}{R_{1}}
= \frac{n_{1}}{n_{\ell}}\,\theta_{1}
  + \frac{n_{\ell}-n_{1}}{n_{\ell}}\,\frac{y_{1}}{R_{1}}.
\]

La matriz que relaciona $(y_{1},\theta_{1})$ con $(y',\theta')$ quedaria como:
\[
M_{1} =
\begin{pmatrix}
1 & 0 \\[6pt]
\dfrac{n_{\ell}-n_{1}}{n_{\ell}\,R_{1}}
& \dfrac{n_{1}}{n_{\ell}}
\end{pmatrix}.
\]

\subsection{Matriz $M_{2}$: traslación en el espesor}

Nuestra $M_{2}$ describe la propagación del rayo a por medio del espesor central $t$ en  $n_{\ell}$. 
\[
y'' = y' + t\,\tan\theta' \approx y' + t\,\theta',
\qquad
\theta'' = \theta'.
\]
Sin cambios ni en el índice ni curvatura, quedando
\[
M_{2} =
\begin{pmatrix}
1 & t \\[6pt]
0 & 1
\end{pmatrix},
\]


\subsection{Matriz $M_{3}$: refracción en la segunda superficie}

$M_{3}$ modela la refracción de el rayo cuando este va hacia el material con índice $n_{\ell}$ hasta $n_{2}$, con radio $R_{2}$ de la superficie.
De manera similar a $M_{2}$ la ley de Snell paraxial produce  
\[
n_{\ell}\,\alpha'' = n_{2}\,\alpha_{2}
\quad\Longrightarrow\quad
\alpha_{2} = \frac{n_{\ell}}{n_{2}}\Bigl(\theta'' + \frac{y''}{R_{2}}\Bigr).
\]
quedando similares, a diferencia de los signos
\[
\theta_{2} = \alpha_{2} - \frac{y''}{R_{2}}
= \frac{n_{\ell}}{n_{2}}\,\theta''
  - \frac{n_{\ell}-n_{2}}{n_{2}}\,\frac{y''}{R_{2}},
\]
Como la altura final obtenida $y_{2}=y''$. Entonces:
\[
M_{3} =
\begin{pmatrix}
1 & 0 \\[6pt]
-\,\dfrac{n_{\ell}-n_{2}}{n_{2}\,R_{2}}
& \dfrac{n_{\ell}}{n_{2}}
\end{pmatrix}.
\]






La matriz total se obtiene por la composición
\[
M=M_{3}M_{2}M_{1}=
\begin{pmatrix}A& B\\C& D\end{pmatrix}.
\]
Definiendo
\[
M_{1}=\begin{pmatrix}1&0\\m_{1}&k_{1}\end{pmatrix},
\quad
M_{2}=\begin{pmatrix}1&t\\0&1\end{pmatrix},
\quad
M_{3}=\begin{pmatrix}1&0\\m_{3}&k_{3}\end{pmatrix},
\]
con
\[
m_{1}=\frac{n_{\ell}-n_{1}}{n_{\ell}R_{1}},
\quad k_{1}=\frac{n_{1}}{n_{\ell}},
\quad m_{3}=-\frac{n_{\ell}-n_{2}}{n_{2}R_{2}},
\quad k_{3}=\frac{n_{\ell}}{n_{2}},
\]
el producto directo da:
\[
M_{2}M_{1}=\begin{pmatrix}1+t\,m_{1}&t\,k_{1}\\m_{1}&k_{1}\end{pmatrix},
\quad
M=M_{3}(M_{2}M_{1})=
\begin{pmatrix}A&B\\C&D\end{pmatrix},
\]
y los elementos son
\[
\begin{aligned}
A&=1+t\,m_{1},\\
B&=t\,k_{1},\\
C&=m_{3}(1+t\,m_{1})+k_{3}m_{1},\\
D&=m_{3}(t\,k_{1})+k_{3}k_{1}.
\end{aligned}
\]
Sustituyendo y simplificando:
\[
\begin{aligned}
A&=1+\frac{n_{\ell}-n_{1}}{n_{\ell}R_{1}}\,t,\\
B&=\frac{n_{1}}{n_{\ell}}\,t,\\
C&=-\frac{n_{\ell}-n_{2}}{n_{2}R_{2}}\Bigl(1+\frac{n_{\ell}-n_{1}}{n_{\ell}R_{1}}\,t\Bigr)
+\frac{n_{\ell}-n_{1}}{n_{2}R_{1}},\\
D&=-\frac{n_{\ell}-n_{2}}{n_{2}R_{2}}\,t\,\frac{n_{1}}{n_{\ell}}+\frac{n_{1}}{n_{2}}.
\end{aligned}
\]



En óptica paraxial, cualquier rayo se describe por
\[
\begin{pmatrix}y_2\\\theta_2\end{pmatrix}
=
\begin{pmatrix}A & B\\C & D\end{pmatrix}
\begin{pmatrix}y_1\\\theta_1\end{pmatrix}.
\]

\subsection{Focal frontal \(f_{1}\)}

Partimos de la matriz total del sistema M, para definir el foco frontal \(F_{1}\) imponemos la condición de que un rayo que parte de la altura \(y_{1}\) con ángulo  
\[
\theta_{1} \;=\; -\,\frac{y_{1}}{f_{1}}
\]
emerja paralelo al eje óptico, es decir \(\theta_{2}=0\). Del segundo componente del producto matricial se extrae  
\[
\theta_{2} \;=\; C\,y_{1} \;+\; D\,\theta_{1}\,,
\]
y al sustituir \(\theta_{1}=-y_{1}/f_{1}\) se obtiene  
\[
0 = C\,y_{1} - D\,\frac{y_{1}}{f_{1}}
\quad\Longrightarrow\quad
\frac{1}{f_{1}} \;=\;\frac{C}{D}.
\]
Se factoriza el índice de entrada para escribir  
\[
C = -\,\frac{n_{1}}{n_{2}}\,\frac{1}{f_{1}}
\quad\Longrightarrow\quad
\frac{1}{f_{1}} = -\,\frac{n_{2}}{n_{1}}\,C.
\]
Por otra parte, la combinación de las matrices lleva a la expresión  
\[
C \;=\;
\frac{n_{\ell}-n_{1}}{n_{1}\,R_{1}}
\;+\;
\frac{n_{2}-n_{\ell}}{n_{2}\,R_{2}}
\;+\;
\frac{(n_{\ell}-n_{1})(n_{2}-n_{\ell})}{n_{1}\,n_{2}}\,
\frac{t}{R_{1}R_{2}},
\]
de modo que al sustituir en \(\,1/f_{1}=-\tfrac{n_{2}}{n_{1}}\,C\)  
\[
\boxed{
\frac{1}{f_{1}}
=
\frac{n_{\ell}-n_{1}}{n_{1}\,R_{1}}
+
\frac{n_{2}-n_{\ell}}{n_{2}\,R_{2}}
+
\frac{(n_{\ell}-n_{1})(n_{2}-n_{\ell})}{n_{1}\,n_{2}}
\,\frac{t}{R_{1}R_{2}}.
}
\]



\subsection{Foco trasero $f_{2}$}

Con la formulación de Newton definimos
\[
x = z_{1} - f_{1},\quad x' = z_{2} - f_{2},\quad x\,x' = f_{1}\,f_{2}.
\]
Si invertimos el sentido de propagación (intercambiando $n_{1}\leftrightarrow n_{2}$), la expresión para $1/f_{2}$ es idéntica a la de $1/f_{1}$ salvo un signo por convención de ejes. De ahí surge de forma inmediata
\[
\boxed{%
f_{2} = -\,\frac{n_{2}}{n_{1}}\,f_{1}.
}
\]
Alternativamente, colocando el objeto en el plano focal trasero invertido ($x=0$), la condición $x'\to\infty$ también conduce al mismo factor $-\,n_{2}/n_{1}$.



\subsection{Plano principal frontal $r$}

Definimos $P$ como el punto sobre el eje (a distancia $p$ del vértice de entrada) tal que un rayo que parte de él con altura $y_1$ y pendiente $\theta_1$ emerja paralelo ($\theta_2=0$).  

Como $\theta_1\approx y_1/p$, se cumple
\[
y_1 = p\,\theta_1.
\]
La condición $\theta_2=0$ conduce a
\[
0 = C\,y_1 + D\,\theta_1 = (C\,p + D)\,\theta_1
\;\Longrightarrow\;
p = -\frac{D}{C}.
\]
Usando la convención $p = D/C$, el plano principal frontal $H_1$, a distancia $r$ del vértice, queda
\[
\boxed{%
r = p - f_{1} = \frac{D}{C} \;-\; f_{1}.
}
\]

\subsection{Plano principal trasero $s$}

Consideremos un rayo que entra paralelo al eje ($\theta_1=0$). Al atravesar la lente, sus valores de salida son
\[
y_2 = A\,y_1,
\qquad
\theta_2 = C\,y_1.
\]
Si prolongamos este rayo hacia atrás desde el segundo vértice, el punto de intersección con el eje se sitúa a
\[
q = \frac{y_2}{\theta_2} = \frac{A}{C}.
\]
Como el foco trasero $F_2$ está a la distancia $f_2$ del vértice, el plano principal trasero $H_2$ se encuentra en
\[
\boxed{s = q \;-\; f_{2} = \frac{A}{C} \;-\; f_{2}.}
\]

\subsection{Plano nodal frontal $v$}

El plano nodal frontal $N_1$ es aquel desde el cual un rayo dirigido hacia él emerge manteniendo su ángulo. Primero imponemos que al llegar paralelo ($\theta_2=0$) se cumpla
\[
\theta_{2} = C\,y_{1} + D\,\theta_{1} = 0,
\]
y si suponemos $\theta_{1}\approx -y_{1}/v$, obtenemos
\[
v = \frac{D}{C}.
\]
Luego, para asegurar que la altura de salida sea cero, sustituimos $\theta_{1}=-y_{1}/f_{1}$ en
\[
y_{2} = A\,y_{1} + B\,\theta_{1},
\]
lo que da $B = A\,f_{1}$. Con ambas condiciones alcanzamos la fórmula final:
\[
\boxed{%
v \;=\;\frac{A\,f_{1} - B}{C}.
}
\]





\subsection{Plano nodal trasero $w$}

El plano nodal trasero $N_{2}$, medido desde el vértice de salida, es el punto donde un rayo paralelo ($\theta_{1}=0$) mantiene su inclinación al salir. 

Al atravesar la lente:
\[
y_{2} = A\,y_{1}, 
\qquad
\theta_{2} = C\,y_{1}.
\]
Prolongando el rayo hacia atrás hasta que recupere altura cero:
\[
0 = A\,y_{1} - w\,(C\,y_{1})
\;\Longrightarrow\;
w = \frac{A}{C}.
\]
Por último, descontamos la distancia al foco trasero $f_{2}$:
\[
\boxed{%
w = \frac{A}{C} \;-\; f_{2}.
}
\]

%%%%%%%%%%%%%%%%%%%%%%%%%%%%%%%%%%%%%%%%%%%%%%%%%%%%%%%%%%%%%%%%%%%%%%%%
\subsection{Relación de Newton}

Usando las coordenadas de Newton,
\[
x = z_{1} - f_{1},\quad x' = z_{2} - f_{2},\quad x\,x' = f_{1}\,f_{2},
\]
sustituimos para obtener
\[
(z_{1}-f_{1})(z_{2}-f_{2}) = f_{1}f_{2}
\;\Longrightarrow\;
z_{1}z_{2} - f_{1}z_{2} - f_{2}z_{1} = 0.
\]
Dividiendo por $z_{1}z_{2}$ y reorganizando términos resulta
\[
\boxed{%
-\,\frac{f_{1}}{z_{1}} + \frac{f_{2}}{z_{2}} = 1.
}
\]

%%%%%%%%%%%%%%%%%%%%%%%%%%%%%%%%%%%%%%

\subsection{Aumento lateral $m$}

Introducimos traslaciones desde $H_1$ y hasta $H_2$:
\[
T_{\rm in}=\begin{pmatrix}1 & z_1\\0 & 1\end{pmatrix},\quad
T_{\rm out}=\begin{pmatrix}1 & z_2\\0 & 1\end{pmatrix}.
\]
La matriz conjugada
\[
M' = T_{\rm out}\,M\,T_{\rm in}
= \begin{pmatrix}A' & B'\\C' & D'\end{pmatrix}
\]
tiene
\[
B' = z_1A + B + z_2\,(z_1C + D).
\]
Imponiendo $B'=0$ (planos conjugados) hallamos
\[
z_2 = -\frac{z_1A + B}{z_1C + D},
\]
y el elemento
\[
A' = A + C\,z_2
\]
es la razón de alturas, esto es
\[
\boxed{%
m = A + C\,z_2.
}
\]
Aplicando la forma de Gauss con $n_1,n_2$,
\[
z_2 = -\frac{n_1}{n_2}\frac{A\,z_1+B}{C\,z_1+D}
\;\Longrightarrow\;
m = -\frac{n_1}{n_2}\frac{z_2}{z_1}.
\]

%%%%%%%%%%%%%%%%%%%%%%%%%%%%%%%%%%%%%%%%%%%%%%%%%%%%%%%%%%%%%%%%%%%%%%%%

\subsection{Caso $n_1=n_2$: lente delgada}

En aire ($n_1=n_2$) y usando $D=Cf_1$, la condición $B'=0$ conduce a
\[
A\,z_1 + B + z_2(C\,z_1+D)=0
\;\Longrightarrow\;
\frac{1}{z_1} + \frac{1}{z_2} = \frac{1}{f_1}.
\]
El elemento $A'=A+Cz_2$ sigue siendo la relación de alturas, de modo que
\[
m = A + C\,z_2
\quad\Longrightarrow\quad
\boxed{m = -\frac{z_2}{z_1}.}
\]
\clearpage
%%%%%%%%%%%%%%%%%%%%%%%%%%%%%%%%%%%%%%%%%%%%%%%%%%%%%%%%%%%%%%%%%%%%%%%%%%%%%%%%%%%%%%%%%%%%%%%%%%%%%%%%%%%%%%%%%%%%%%%%%%%%%%%%%%%%%%%%%%%%%%%%%%%%%%%%%%%%%%%%%%%%%%%%%%%%%%%%%%%%%%%%%%%%%%%%%%%%%%%

\end{document}
